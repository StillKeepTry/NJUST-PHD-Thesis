\chapter{第四章}
\label{chap:chap4}
模式识别是将采集到的指纹数据和字典进行匹配,Ma等\cite{mrf}使用了模板匹配的方法来进行参数图的重建。记$X=\{x_n\in \mathbb{C}^L\}, n=1,...,N$为采集到的指纹数据,$\mathcal{D}=\{d_k\in \mathbb{C}^L\},k=1,...,K$为生成的字典。那么模板匹配即为从从字典$\mathcal{D}$中选取和$x_n$内积最大的元素:
	\begin{equation}
	\hat{k}_n = \argmax_k \left|\left\langle \mathbf{d}_k,\mathbf{x}_n \right\rangle \right|.
	\end{equation}
并且质子密度也可以同时计算:
	\begin{equation}
	\hat{\rho}_n=\left|\langle \mathbf{d}_{\hat{k}_n},\mathbf{x}_n \rangle\right|.
	\end{equation}

\begin{algorithm}
	\caption{snapMRF生成字典与匹配详细流程。}
	\label{alg:snapMRF}
	\begin{algorithmic}
		\INDSTATE[-1.25] \textbf{输入:}\texttt{*d\_mrf},\texttt{*d\_params},\texttt{*d\_img}
		\INDSTATE[-1.25] \textbf{输出:}\texttt{*d\_atoms},\texttt{*d\_maps}
		\STATE 01:从CSV文件中读取MRF序列信息,存入\texttt{*d\_mrf};
		\STATE 02:从命令行输入字典参数信息,存入\texttt{*d\_params};
		\STATE 03:初始化状态矩阵\texttt{*d\_w};
		\STATE 04:\textbf{迭代}:从第1个时刻到第$L$个时刻,并行计算所有体素的信号
		\STATE 05:\qquad 使用函数\texttt{fill\_transition\_matrix()}构造转移矩阵$T$;
		\STATE 06:\qquad 使用函数\texttt{apply\_rf\_pulse()}将射频场作用在\texttt{*d\_w}上;
		\STATE 07:\qquad 使用函数\texttt{decay\_signal()}将$T_1$和$T_2$衰减作用在\texttt{*d\_w}上;
		\STATE 08:\qquad 使用函数\texttt{save\_atoms()}将原子的信号保存在\texttt{*d\_atoms}中;
		\STATE 09:\qquad 使用函数\texttt{dephase\_gradients()}将梯度场作用在\texttt{*d\_w}上;
		\STATE 10:\qquad 使用函数\texttt{decay\_signal()}将$T_1$和$T_2$衰减作用在\texttt{*d\_w}上;
		\STATE 11:\textbf{终止迭代};
		\STATE 12:释放\texttt{*d\_w};
		\STATE 13:从RawArray文件中读取指纹数据,存入\texttt{*d\_img};
		\STATE 14:计算剩余显存大小,并根据剩余显存,将\texttt{*d\_img}分为$G$组;
		\STATE 15:\textbf{迭代}:从第1组到第$G$组,在每一组内并行计算所有体素的参数
		\STATE 16:\qquad 使用函数\texttt{MRF\_minimatch()}进行匹配;
		\STATE 17:\qquad 使用函数\texttt{generate\_maps()}生成参数图;
		\STATE 18:\textbf{终止迭代};
		\STATE 19:将\texttt{*d\_atoms}和\texttt{*d\_maps}保存为RawArray文件;
		\STATE 20:释放所有显存和内存。
	\end{algorithmic}
\end{algorithm}











